\documentclass{exam}
\usepackage{amsmath}
\usepackage{amsfonts}
\usepackage{hhline}
\usepackage{url}
\usepackage{hyperref}
\usepackage{bussproofs}

\title{CS251 homework 3}
\author{name:\underline{\hspace{2in}}}
\date{Due: 10/29/19}

% these are several symbols that I redefined to make formulas easier to read
\def\land{\wedge}           % /\
\def\lor{\vee}              % \/
\def\lnot{\neg}             % -
\def\liff{\leftrightarrow}  % <->
\def\xor{\oplus}            % (+)
\def\T{\top}                % T
\def\F{\bot}                % _|_

\def\proves{\vdash}
\newcommand{\bra}[1]{\langle#1|}
\newcommand{\ket}[1]{|#1\rangle}
\newcommand{\braket}[2]{\langle#1|#2\rangle}

\def\R{\mathbb{R}}
\def\C{\mathbb{C}}


\begin{document}
\maketitle

% these are commands that use the bussproofs package.
% they let you write tree style proofs.
% you can get the package here:  https://www.math.ucsd.edu/~sbuss/ResearchWeb/bussproofs/bussproofs.sty

\newcommand{\premise}    [1]{\AxiomC{#1}}
\newcommand{\assumption} [1]{\AxiomC{[#1]}}
\newcommand{\andI}       [1]{\RightLabel{$\land I$}   \BinaryInfC{#1} }
\newcommand{\andEL}      [1]{\RightLabel{$\land E1$}  \UnaryInfC{#1}  }
\newcommand{\andER}      [1]{\RightLabel{$\land E2$}  \UnaryInfC{#1}  }
\newcommand{\orIL}       [1]{\RightLabel{$\lor I1$}   \UnaryInfC{#1}  }
\newcommand{\orIR}       [1]{\RightLabel{$\lor I2$}   \UnaryInfC{#1}  }
\newcommand{\orE}        [1]{\RightLabel{$\lor E$}    \TrinaryInfC{#1}}
\newcommand{\arrowI}     [1]{\RightLabel{$\to I$}     \BinaryInfC{#1} }
\newcommand{\arrowE}     [1]{\RightLabel{$\to E$}     \BinaryInfC{#1} }
\newcommand{\notE}       [1]{\RightLabel{$\lnot E$}   \BinaryInfC{#1} }
\newcommand{\notI}       [1]{\RightLabel{$\lnot I$}   \UnaryInfC{#1}  }
\newcommand{\FE}         [1]{\RightLabel{$\F E$}      \UnaryInfC{#1}  }
\newcommand{\TI}         [1]{\RightLabel{$\T I$}      \UnaryInfC{#1}  }
\newcommand{\LEM}        [1]{\RightLabel{$LEM$}       \UnaryInfC{#1}  }
\newcommand{\DLL}        [1]{\RightLabel{$DL1$}       \UnaryInfC{#1}  }
\newcommand{\DLR}        [1]{\RightLabel{$DL2$}       \UnaryInfC{#1}  }
\newcommand{\notNotE}    [1]{\RightLabel{$\lnot \lnot E$} \UnaryInfC{#1} }
\newcommand{\ForI}       [1]{\RightLabel{$\forall I$} \BinaryInfC{#1}}
\newcommand{\ForE}       [1]{\RightLabel{$\forall E$} \UnaryInfC{#1} }
\newcommand{\ExI}        [1]{\RightLabel{$\exists I$} \UnaryInfC{#1} }
\newcommand{\ExE}        [1]{\RightLabel{$\exists E$} \BinaryInfC{#1}}
\newcommand{\refl}       [1]{\premise{$\ $}\RightLabel{$Refl$}      \UnaryInfC{#1}   }
\newcommand{\eqEL}       [1]{\RightLabel{$=E1$}                     \BinaryInfC{#1}  }
\newcommand{\eqER}       [1]{\RightLabel{$=E2$}                     \BinaryInfC{#1}  }

% we can write proofs by just writing the inference rules
% for example we can write the proof

% A /\ B       A /\ B
% ------/\E1  --------/\E2
%   B            A
%   -------------- /\I
%       B /\ A

% as
%
% \begin{prooftree}
%   \premise{$A \land B$}
%   \andER{$B$}
%
%   \premise{$A \land B$}
%   \andEL{$A$}
%
%   \andI{$B \land A$}
% \end{prooftree}

\begin{questions}


\question
Prove the following.\\
use the homework4.py file to verify that the proofs are correct.
\begin{parts}
  \part $\forall x. \forall y. P(x,y) \proves \forall y. \forall x. P(x,y)$
        \vspace{5cm}

  \part $\forall x. \lnot P(x) \proves \lnot \exists x. P(x)$
        \vspace{5cm}

  \part $\exists x. \lnot P(x) \proves \lnot \forall x. P(x)$
        \vspace{5cm}

  \part $\lnot \exists x. P(x) \proves \forall x. \lnot P(x)$
        \vspace{5cm}

  \part $\forall x. P(x) \to Q \proves (\exists x. P(x)) \to Q$
        \vspace{5cm}
\end{parts}


\question
Prove the following using equality.\\
You don't need to code these in python.
\begin{parts}
  \part $x = y \proves y = x$
        \vspace{3cm}

  \part $x = y, y = z \proves x = z$
        \vspace{3cm}

\pagebreak

  \part $\forall y. y = 0 \lor (\exists z. y = z+1),$\\
        $\forall x. x + 0 = x, $\\
        $\forall x. x = 0 + x, $\\
        $\forall x y. x + (y + 1) = (y + 1) + x,$\\
        $\proves \forall n m. n + m = m + n$
        \vspace{9cm}
\end{parts}

\pagebreak

\question

It turns out that you don't have to actually understand something in order to prove it.\\
$\ $\\
To demonstrate this:\\
Prove the No Cloning theorem from Quantum Mechanics.\\
If you want an explanation of what you're actually proving,\\
Minute Physics did a video on YouTube.\\
\url{https://www.youtube.com/watch?v=owPC60Ue0BE}\\
$\ $\\
$\forall (U\in \C^{n\times n}) (a\in \C^n) (b\in \C^n). U(\braket a b) = \braket a b$,\\
$\forall (U\in \C^{n\times n}) (a\in \C^n) (b\in \C^n). U(a b) = U(a) U(b)$,\\
$\forall a\in \R.\ (a\cdot a = a) \to (a = 0 \lor a = 1)$,\\
$\forall (a\in \C^n) (b\in \C^n). \braket a b = (\bra a \otimes\bra 0)(\ket b \otimes \ket 0)$,\\
$\forall (a\in \C^n) (b\in \C^n). (\bra a \otimes\bra a)(\ket b \otimes \ket b) = \braket a b \cdot \braket a b $,\\
$\proves$\\
$((U(\ket a \otimes \ket 0) = \ket a \otimes\ket a) \land 
(U(\ket b \otimes\ket 0)  = \ket b \otimes\ket b)) \to
\braket a b = 0 \lor \braket a b = 1$\\



\end{questions}
\end{document}
