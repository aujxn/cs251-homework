\documentclass{article}

\usepackage{fancyhdr}
\usepackage{extramarks}
\usepackage{amsmath}
\usepackage{amssymb}
\usepackage{amsthm}
\usepackage{amsfonts}
\usepackage{tikz}
\usepackage[plain]{algorithm}
\usepackage{algpseudocode}
\usepackage{bussproofs}
\usepackage{hhline}

\usetikzlibrary{automata,positioning}

%
% Basic Document Settings
%

\topmargin=-0.45in
\evensidemargin=0in
\oddsidemargin=0in
\textwidth=6.5in
\textheight=9.0in
\headsep=0.25in

\linespread{1.1}

\pagestyle{fancy}
\lhead{\hmwkAuthorName}
\chead{\hmwkClass\ (\hmwkClassInstructor\ \hmwkClassTime): \hmwkTitle}
\rhead{\firstxmark}
\lfoot{\lastxmark}
\cfoot{\thepage}

\renewcommand\headrulewidth{0.4pt}
\renewcommand\footrulewidth{0.4pt}

\setlength\parindent{0pt}

%
% Create Problem Sections
%

\newcommand{\enterProblemHeader}[1]{
    \nobreak\extramarks{}{Problem \arabic{#1} continued on next page\ldots}\nobreak{}
    \nobreak\extramarks{Problem \arabic{#1} (continued)}{Problem \arabic{#1} continued on next page\ldots}\nobreak{}
}

\newcommand{\exitProblemHeader}[1]{
    \nobreak\extramarks{Problem \arabic{#1} (continued)}{Problem \arabic{#1} continued on next page\ldots}\nobreak{}
    \stepcounter{#1}
    \nobreak\extramarks{Problem \arabic{#1}}{}\nobreak{}
}

\setcounter{secnumdepth}{0}
\newcounter{partCounter}
\newcounter{homeworkProblemCounter}
\setcounter{homeworkProblemCounter}{1}
\nobreak\extramarks{Problem \arabic{homeworkProblemCounter}}{}\nobreak{}

%
% Homework Problem Environment
%
% This environment takes an optional argument. When given, it will adjust the
% problem counter. This is useful for when the problems given for your
% assignment aren't sequential. See the last 3 problems of this template for an
% example.
%
\newenvironment{homeworkProblem}[1][-1]{
    \ifnum#1>0
        \setcounter{homeworkProblemCounter}{#1}
    \fi
    \section{Problem \arabic{homeworkProblemCounter}}
    \setcounter{partCounter}{1}
    \enterProblemHeader{homeworkProblemCounter}
}{
    \exitProblemHeader{homeworkProblemCounter}
}

%
% Homework Details
%   - Title
%   - Due date
%   - Class
%   - Section/Time
%   - Instructor
%   - Author
%

\newcommand{\hmwkTitle}{Homework \#4}
\newcommand{\hmwkDueDate}{November 5, 2019}
\newcommand{\hmwkClass}{CS251}
\newcommand{\hmwkClassTime}{Section A}
\newcommand{\hmwkClassInstructor}{Steven Libby}
\newcommand{\hmwkAuthorName}{Austen Nelson}

%
% Title Page
%

\title{
    \vspace{2in}
    \textmd{\textbf{\hmwkClass:\ \hmwkTitle}}\\
    \normalsize\vspace{0.1in}\small{Due\ on\ \hmwkDueDate\ at 2:00pm}\\
    \vspace{0.1in}\large{\textit{\hmwkClassInstructor\ \hmwkClassTime}}
    \vspace{3in}
}

\author{\hmwkAuthorName}
\date{}

\renewcommand{\part}[1]{\textbf{\large Part \Alph{partCounter}}\stepcounter{partCounter}\\}

%
% Various Helper Commands
%

% Useful for algorithms
\newcommand{\alg}[1]{\textsc{\bfseries \footnotesize #1}}

% For derivatives
\newcommand{\deriv}[1]{\frac{\mathrm{d}}{\mathrm{d}x} (#1)}

% For partial derivatives
\newcommand{\pderiv}[2]{\frac{\partial}{\partial #1} (#2)}

% Integral dx
\newcommand{\dx}{\mathrm{d}x}

% Alias for the Solution section header
\newcommand{\solution}{\textbf{\large Solution}}

% Probability commands: Expectation, Variance, Covariance, Bias
\newcommand{\E}{\mathrm{E}}
\newcommand{\Var}{\mathrm{Var}}
\newcommand{\Cov}{\mathrm{Cov}}
\newcommand{\Bias}{\mathrm{Bias}}

\newcommand{\bra}[1]{\langle#1|}
\newcommand{\ket}[1]{|#1\rangle}
\newcommand{\braket}[2]{\langle#1|#2\rangle}
\def\R{\mathbb{R}}
\def\C{\mathbb{C}}
\def\proves{\vdash}

% these are several symbols that I redefined to make formulas easier to read
\def\land{\wedge}           % /\
\def\lor{\vee}              % \/
\def\lnot{\neg}             % -
\def\liff{\leftrightarrow}  % <->
\def\xor{\oplus}            % (+)
\def\T{\top}                % T
\def\F{\bot}                % _|_
\def\proves{\vdash}         % |-


\newcommand{\premise}    [1]{\AxiomC{#1}}
\newcommand{\assumption} [1]{\AxiomC{[#1]}}
\newcommand{\andI}       [1]{\RightLabel{$\land I$} \BinaryInfC{#1} }
\newcommand{\andEL}      [1]{\RightLabel{$\land E1$}\UnaryInfC{#1}  }
\newcommand{\andER}      [1]{\RightLabel{$\land E2$}\UnaryInfC{#1}  }
\newcommand{\EER}        [1]{\RightLabel{$= E2$}\BinaryInfC{#1}  }
\newcommand{\EEL}        [1]{\RightLabel{$= E1$}\BinaryInfC{#1}  }
\newcommand{\orIL}       [1]{\RightLabel{$\lor I1$} \UnaryInfC{#1}  }
\newcommand{\orIR}       [1]{\RightLabel{$\lor I2$} \UnaryInfC{#1}  }
\newcommand{\orE}        [1]{\RightLabel{$\lor E$}  \TrinaryInfC{#1}}
\newcommand{\arrowI}     [1]{\RightLabel{$\to I$}   \BinaryInfC{#1} }
\newcommand{\arrowE}     [1]{\RightLabel{$\to E$}   \BinaryInfC{#1} }
\newcommand{\notE}       [1]{\RightLabel{$\lnot E$} \BinaryInfC{#1} }
\newcommand{\notI}       [1]{\RightLabel{$\lnot I$} \UnaryInfC{#1}  }
\newcommand{\FE}         [1]{\RightLabel{$\F E$}    \UnaryInfC{#1}  }
\newcommand{\TI}         [1]{\RightLabel{$\T I$}    \UnaryInfC{#1}  }
\newcommand{\LEM}        [1]{\RightLabel{$LEM$}     \UnaryInfC{#1}  }
\newcommand{\REFL}       [1]{\RightLabel{$REFL$}     \UnaryInfC{#1}  }
\newcommand{\DLL}        [1]{\RightLabel{$DL1$}     \UnaryInfC{#1}  }
\newcommand{\DLR}        [1]{\RightLabel{$DL2$}     \UnaryInfC{#1}  }
\newcommand{\notNotE}    [1]{\RightLabel{$\lnot \lnot E$} \UnaryInfC{#1} }
\begin{document}

\maketitle

\pagebreak

\begin{homeworkProblem}
   \textbf{Part One}\\
   \begin{prooftree}
      \premise{$x = y$}
      \premise{$\ $}
      \REFL{$y = y$}
      \EER{$y = x$}
   \end{prooftree}
   \textbf{Part Two}\\
   \begin{prooftree}
      \premise{$x = y$}
      \premise{$y = z$}
      \EEL{$x = z$}
   \end{prooftree}
   \textbf{Part Three}\\
        $\forall y. y = 0 \lor (\exists z. y = z+1),$\\
        $\forall x. x + 0 = x, $\\
        $\forall x. x = 0 + x, $\\
        $\forall x y. x + (y + 1) = (y + 1) + x,$\\
        $\proves \forall n m. n + m = m + n$
   \begin{align*}
      & 1. &&\forall y. y = 0 \lor (\exists z. y = z+1)
      && Premise
      \\
      & 2. &&\forall x. x + 0 = x
      && Premise
      \\
      & 3. &&\forall x. x = 0 + x
      && Premise
      \\
      & 4. &&\forall x y. x + (y + 1) = (y + 1) + x
      && Premise
      \\
      & 5. &&[c], [d]
      && Assumptions
      \\
      & 6. &&[d=e+1]
      && Assumption
      \\
      & 7. && c + (e + 1) = (e + 1) + c
      && \forall E, 4
      \\
      & 8. && c + (e + 1) = d + c
      && = E2, 6, 7
      \\
      & 9. && c + d = d + c
      && = E1, 6, 8
      \\
      & 10.&& (d = e + 1) \to (c + d = d + c)
      && \to I, 6-10
      \\
      & 11.&& [d=0]
      && assumption
      \\
      & 12.&& c + 0 = c
      && \forall E, 2
      \\
      & 13.&& c = 0 + c
      && \forall E, 3
      \\
      & 14.&& c = d + c
      && =E2, 11,13
      \\
      & 15.&& c + 0 = d + c
      && =E1, 12,14
      \\
      & 16.&& c + d = d + c
      && =E1, 11, 16
      \\
      & 17.&& (d=0) \to (c + d = d + c)
      && \to I, 11 - 16
      \\
      & 18.&& d = 0 \lor  = e + 1
      && \forall E, 1
      \\
      & 19.&& c + d = d + c
      && \lor E, 18, 17, 10
      \\
      & 20.&& \forall n,m. n+m = m+n
      && \forall I, 5-19
      \\
   \end{align*}
\end{homeworkProblem}
\pagebreak
\begin{homeworkProblem}
$\forall (U\in \C^{n\times n}) (a\in \C^n) (b\in \C^n). U(\braket a b) = \braket a b$,\\
$\forall (U\in \C^{n\times n}) (a\in \C^n) (b\in \C^n). U(a b) = U(a) U(b)$,\\
$\forall a\in \R.\ (a\cdot a = a) \to (a = 0 \lor a = 1)$,\\
$\forall (a\in \C^n) (b\in \C^n). \braket a b = (\bra a \otimes\bra 0)(\ket b \otimes \ket 0)$,\\
$\forall (a\in \C^n) (b\in \C^n). (\bra a \otimes\bra a)(\ket b \otimes \ket b) = \braket a b \cdot \braket a b $,\\
$\proves$\\
$((U(\ket a \otimes \ket 0) = \ket a \otimes\ket a) \land 
(U(\ket b \otimes\ket 0)  = \ket b \otimes\ket b)) \to
\braket a b = 0 \lor \braket a b = 1$\\

   \begin{align*}
      & 1. &&\forall (U\in \C^{n\times n}) (a\in \C^n) (b\in \C^n). U(\braket a b) = \braket a b
      && Premise
      \\
      & 2. &&\forall (U\in \C^{n\times n}) (a\in \C^n) (b\in \C^n). U(a b) = U(a) U(b)
      && Premise
      \\
      & 3. &&\forall a\in \R.\ (a\cdot a = a) \to (a = 0 \lor a = 1)
      && Premise
      \\
      & 4. &&\forall (a\in \C^n) (b\in \C^n). \braket a b = (\bra a \otimes\bra 0)(\ket b \otimes \ket 0)
      && Premise
      \\
      & 5. &&\forall (a\in \C^n) (b\in \C^n). (\bra a \otimes\bra a)(\ket b \otimes \ket b) = \braket a b \cdot \braket a b
      && Premise
      \\
      & 6. &&[((U(\ket a \otimes \ket 0) = \ket a \otimes\ket a) \land (U(\ket b \otimes\ket 0)  = \ket b \otimes\ket b))]
      && Assumption
      \\
      \hline
      & 7. &&U(\ket a \otimes \ket 0) = \ket a \otimes\ket a
      && \land E1, 6
      \\
      & 8. &&U(\ket a \otimes \ket 0) = U(\ket a) \otimes U(\ket 0)
      && \forall E, 2
      \\
      & 9. &&U(\ket a) \otimes U(\ket 0) =  \ket a \otimes\ket a
      && =E1,7,8
      \\
      & 10.&&U(\ket a) =  \ket a
      && \forall E, 1
      \\
      & 11.&&U(\ket 0) =  \ket 0
      && \forall E, 1
      \\
      & 12.&&\ket a \otimes \ket 0 =  \ket a \otimes\ket a
      && =E1*2,9,10,11
      \\
      \hline
      & 13.&&U(\ket b \otimes \ket 0) = \ket b \otimes\ket b
      && \land E2, 6
      \\
      & 14.&&U(\ket b \otimes \ket 0) = U(\ket b) \otimes U(\ket 0)
      && \forall E, 2
      \\
      & 15.&&U(\ket b) \otimes U(\ket 0) =  \ket b \otimes\ket b
      && =E1,13,14
      \\
      & 16.&&U(\ket b) =  \ket b
      && \forall E, 1
      \\
      & 17.&&U(\ket 0) =  \ket 0
      && \forall E, 1
      \\
      & 18.&&\ket b \otimes \ket 0 =  \ket b \otimes\ket b
      && =E1*2,15,16,17
      \\
      \hline
      & 19.&&\braket a b = (\bra a \otimes\bra 0)(\ket b \otimes \ket 0)
      && \forall E,4
      \\
      & 20.&&\braket a b = (\bra a \otimes\bra 0)(\ket b \otimes \ket b)
      && =E1, 18, 19
      \\
      & 21.&& (\bra a \otimes\bra a)(\ket b \otimes \ket b) = \braket a b \cdot \braket a b
      && \forall E,5
      \\
      & 22.&& (\bra a \otimes\bra 0)(\ket b \otimes \ket b) = \braket a b \cdot \braket a b
      && =E1, 12,21
      \\
      & 23.&&\braket a b \cdot \braket a b = \braket a b
      && =E1, 20,22
      \\
      & 24.&&(\braket a b \cdot \braket a b = \braket a b) \to (\braket a b = 0 \lor \braket a b = 1)
      && \forall E, 3
      \\
      & 25.&&\braket a b = 0 \lor \braket a b = 1
      && \to E, 23,24
      \\
      & 26.&&((U(\ket a \otimes \ket 0) = \ket a \otimes\ket a) \land 
      (U(\ket b \otimes\ket 0)  = \ket b \otimes\ket b)) \to
      \braket a b = 0 \lor \braket a b = 1
      && \to I,6-25
      \\
   \end{align*}
\end{homeworkProblem}
\end{document}
