\documentclass{exam}
\usepackage{amsmath}
\usepackage{hhline}
\usepackage{url}

\title{CS251 homework 1}
\author{name:\underline{\hspace{2in}}}
\date{Due: 10/8/19}

% these are several symbols that I redefined to make formulas easier to read
\def\land{\wedge}           % /\
\def\lor{\vee}              % \/
\def\lnot{\neg}             % -
\def\liff{\leftrightarrow}  % <->
\def\xor{\oplus}            % (+)
\def\T{\top}                % T
\def\F{\bot}                % _|_


\begin{document}
\maketitle


\begin{questions}

%%%%%%%%%%%%%%%%%%%%%%%%%%%%%%%%%%%%%%%%%%%%%%%%%%%%%%%%%%%%%%%%%%%%%%%%%%%%%%%%%
% Question 1
%%%%%%%%%%%%%%%%%%%%%%%%%%%%%%%%%%%%%%%%%%%%%%%%%%%%%%%%%%%%%%%%%%%%%%%%%%%%%%%%%

\question
Define variables, and write the following sentences as logical statements.\\

\begin{itemize}
    \item If it's not cloudy, then it's not raining. \vspace{3cm}
    \item If we won the big game, then either we scored more points, or the other team didn't show up. \vspace{3cm}
    \item This is a sentence. \vspace{3cm}
    \item If you don't study for tests, then you won't pass the class.\vspace{3cm}
    \item A graph is planer if it contains neither a minor of $K_{3,3}$ nor $K_5$.\vspace{3cm}
\end{itemize}


%%%%%%%%%%%%%%%%%%%%%%%%%%%%%%%%%%%%%%%%%%%%%%%%%%%%%%%%%%%%%%%%%%%%%%%%%%%%%%%%%
% Question 2
%%%%%%%%%%%%%%%%%%%%%%%%%%%%%%%%%%%%%%%%%%%%%%%%%%%%%%%%%%%%%%%%%%%%%%%%%%%%%%%%%
\question

Draw truth tables for the following formulas.\\

% I used the following code to draw a truth table
% There are other ways, but this is probably the simplest.
% Here array means we're making a 2d array on the page
% |cc|c| means we'll have 3 column with vertical bars on the outside, 
% and one before the last column
% Each column is separates by &, and each line ends with \\.
% Finally \hline makes a horizontal line across the entire table.
%
% I've given you guys the outline of the truth table, you just need to fill in the values

%  \begin{array}{|cc|c|}
%    \hline
%    a  & b  & a \land b \\
%    \hline
%    \T & \T &    \T     \\
%    \T & \F &    \F     \\
%    \F & \T &    \F     \\
%    \F & \F &    \F     \\
%    \hline
%  \end{array} 

\noindent
$a \xor b$
$$
\begin{array}{|c|c||c|}
  \hline
  a  & b  & a \xor b \\
  \hline{|=|=||=|}
     &    &          \\
  \hline
     &    &          \\
  \hline
     &    &          \\
  \hline
     &    &          \\
  \hline
\end{array} 
$$
$\lnot (\lnot a)$
$$
\begin{array}{|c||c|}
  \hline
  a  & \lnot (\lnot a) \\
  \hhline{|=||=|}
     &                 \\
  \hline
     &                 \\
  \hline
\end{array} 
$$
$\lnot b \to \lnot a$
$$
\begin{array}{|c|c||c|}
  \hline
  a  & b  & \lnot a \to \lnot b   \\
  \hhline{|=|=||=|}
     &    &                       \\
  \hline
     &    &                       \\
  \hline
     &    &                       \\
  \hline
     &    &                       \\
  \hline
\end{array} 
$$
$\lnot a \land \lnot b$
$$
\begin{array}{|c|c||c|}
  \hline
  a  & b  & \lnot a \land \lnot b \\
  \hhline{|=|=||=|}
     &    &                       \\
  \hline
     &    &                       \\
  \hline
     &    &                       \\
  \hline
     &    &                       \\
  \hline
\end{array} 
$$
$a \liff (b \liff c)$
$$
\begin{array}{|c|c|c||c|}
  \hline
  a  & b  & c  & a \liff (b \liff c) \\
  \hhline{|=|=|=||=|}
     &    &    &                     \\
  \hline
     &    &    &                     \\
  \hline
     &    &    &                     \\
  \hline
     &    &    &                     \\
  \hline
     &    &    &                     \\
  \hline
     &    &    &                     \\
  \hline
     &    &    &                     \\
  \hline
     &    &    &                     \\
  \hline
\end{array} 
$$
$(a \lor c) \land (b \lor c)$
$$
\begin{array}{|c|c|c||c|}
  \hline
  a  & b  & c  & (a \lor c) \land (b \lor c) \\
  \hhline{|=|=|=||=|}
     &    &    &                             \\
  \hline
     &    &    &                             \\
  \hline
     &    &    &                             \\
  \hline
     &    &    &                             \\
  \hline
     &    &    &                             \\
  \hline
     &    &    &                             \\
  \hline
     &    &    &                             \\
  \hline
     &    &    &                             \\
  \hline
\end{array} 
$$

\pagebreak
\question

%%%%%%%%%%%%%%%%%%%%%%%%%%%%%%%%%%%%%%%%%%%%%%%%%%%%%%%%%%%%%%%%%%%%%%%%%%%%%%%%%
% Question 3
%%%%%%%%%%%%%%%%%%%%%%%%%%%%%%%%%%%%%%%%%%%%%%%%%%%%%%%%%%%%%%%%%%%%%%%%%%%%%%%%%

Reduce the following to the shortest form.\\
Determine if it's satisfiable, a tautology, or neither.
\begin{itemize}
    \item $(a \land \lnot b) \lor \lnot (\lnot a \lor b)$: \\
        \vspace{4cm}
    \item $a \land b \equiv \lnot (\lnot a \land \lnot b)$: \\ 
        \vspace{4cm}
    \item $a \land b \equiv \lnot (\lnot a \lor \lnot b)$: \\ 
        \vspace{4cm}
    \item $a \land (b \lor c) \to a \land (b \land c)$: \\ 
        \vspace{4cm}
\end{itemize}


\pagebreak

%%%%%%%%%%%%%%%%%%%%%%%%%%%%%%%%%%%%%%%%%%%%%%%%%%%%%%%%%%%%%%%%%%%%%%%%%%%%%%%%%
% Question 4
%%%%%%%%%%%%%%%%%%%%%%%%%%%%%%%%%%%%%%%%%%%%%%%%%%%%%%%%%%%%%%%%%%%%%%%%%%%%%%%%%

\question
Draw ASTs for the following boolean expressions
\begin{itemize}
    \item $A \land B \to B \land C$ \vspace{4cm}
    \item $A \to B \land B \to C$ \vspace{4cm}
    \item $\lnot A \land B \lor C \to D$ \vspace{4cm}
    \item $\lnot \lnot A \lor \lnot \lnot B$ \vspace{4cm}
    \item $(A \to B) \lor (B \to A)$ \vspace{4cm}
\end{itemize}


\question

Get the code for representing boolean expresions here:
\url{https://github.com/slibby05/prop.git}.

You'll notice that it doesn't work because the eval method isn't defined.
You need to fix that.

\question

In the file hw1.py make the expressions from problem 4.\\
$\ $\\
Turn in hw1.py and AST.py to D2L.

\end{questions}

\end{document}
