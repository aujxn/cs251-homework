\documentclass{article}

\usepackage{fancyhdr}
\usepackage{extramarks}
\usepackage{amsmath}
\usepackage{amsthm}
\usepackage{amsfonts}
\usepackage{tikz}
\usepackage[plain]{algorithm}
\usepackage{algpseudocode}
\usepackage{fancybox}
\usepackage{bussproofs}
\usepackage{hhline}

\usetikzlibrary{automata,positioning}

%
% Basic Document Settings
%

\topmargin=-0.45in
\evensidemargin=0in
\oddsidemargin=0in
\textwidth=6.5in
\textheight=9.0in
\headsep=0.25in

\linespread{1.1}

\pagestyle{fancy}
\lhead{\hmwkAuthorName}
\chead{\hmwkClass\ (\hmwkClassInstructor\ \hmwkClassTime): \hmwkTitle}
\rhead{\firstxmark}
\lfoot{\lastxmark}
\cfoot{\thepage}

\renewcommand\headrulewidth{0.4pt}
\renewcommand\footrulewidth{0.4pt}

\setlength\parindent{0pt}

%
% Create Problem Sections
%

\newcommand{\enterProblemHeader}[1]{
    \nobreak\extramarks{}{Problem \arabic{#1} continued on next page\ldots}\nobreak{}
    \nobreak\extramarks{Problem \arabic{#1} (continued)}{Problem \arabic{#1} continued on next page\ldots}\nobreak{}
}

\newcommand{\exitProblemHeader}[1]{
    \nobreak\extramarks{Problem \arabic{#1} (continued)}{Problem \arabic{#1} continued on next page\ldots}\nobreak{}
    \stepcounter{#1}
    \nobreak\extramarks{Problem \arabic{#1}}{}\nobreak{}
}

\setcounter{secnumdepth}{0}
\newcounter{partCounter}
\newcounter{homeworkProblemCounter}
\setcounter{homeworkProblemCounter}{1}
\nobreak\extramarks{Problem \arabic{homeworkProblemCounter}}{}\nobreak{}

%
% Homework Problem Environment
%
% This environment takes an optional argument. When given, it will adjust the
% problem counter. This is useful for when the problems given for your
% assignment aren't sequential. See the last 3 problems of this template for an
% example.
%
\newenvironment{homeworkProblem}[1][-1]{
    \ifnum#1>0
        \setcounter{homeworkProblemCounter}{#1}
    \fi
    \section{Problem \arabic{homeworkProblemCounter}}
    \setcounter{partCounter}{1}
    \enterProblemHeader{homeworkProblemCounter}
}{
    \exitProblemHeader{homeworkProblemCounter}
}

%
% Homework Details
%   - Title
%   - Due date
%   - Class
%   - Section/Time
%   - Instructor
%   - Author
%

\newcommand{\hmwkTitle}{Homework\ \#1}
\newcommand{\hmwkDueDate}{October 8, 2019}
\newcommand{\hmwkClass}{CS251}
\newcommand{\hmwkClassTime}{Section A}
\newcommand{\hmwkClassInstructor}{Steven Libby}
\newcommand{\hmwkAuthorName}{\textbf{Austen Nelson}}

%
% Title Page
%

\title{
    \vspace{2in}
    \textmd{\textbf{\hmwkClass:\ \hmwkTitle}}\\
    \normalsize\vspace{0.1in}\small{Due\ on\ \hmwkDueDate\ at 2:00pm}\\
    \vspace{0.1in}\large{\textit{\hmwkClassInstructor\ \hmwkClassTime}}
    \vspace{3in}
}

\author{\hmwkAuthorName}
\date{}

\renewcommand{\part}[1]{\textbf{\large Part \Alph{partCounter}}\stepcounter{partCounter}\\}

%
% Various Helper Commands
%

% Useful for algorithms
\newcommand{\alg}[1]{\textsc{\bfseries \footnotesize #1}}

% For derivatives
\newcommand{\deriv}[1]{\frac{\mathrm{d}}{\mathrm{d}x} (#1)}

% For partial derivatives
\newcommand{\pderiv}[2]{\frac{\partial}{\partial #1} (#2)}

% Integral dx
\newcommand{\dx}{\mathrm{d}x}

% Alias for the Solution section header
\newcommand{\solution}{\textbf{\large Solution}}

% Probability commands: Expectation, Variance, Covariance, Bias
\newcommand{\E}{\mathrm{E}}
\newcommand{\Var}{\mathrm{Var}}
\newcommand{\Cov}{\mathrm{Cov}}
\newcommand{\Bias}{\mathrm{Bias}}

% these are several symbols that I redefined to make formulas easier to read
\def\land{\wedge}           % /\
\def\lor{\vee}              % \/
\def\lnot{\neg}             % -
\def\liff{\leftrightarrow}  % <->
\def\xor{\oplus}            % (+)
\def\T{\top}                % T
\def\F{\bot}                % _|_
\def\proves{\vdash}         % |-


\newcommand{\premise}    [1]{\AxiomC{#1}}
\newcommand{\assumption} [1]{\AxiomC{[#1]}}
\newcommand{\andI}       [1]{\RightLabel{$\land I$} \BinaryInfC{#1} }
\newcommand{\andEL}      [1]{\RightLabel{$\land E1$}\UnaryInfC{#1}  }
\newcommand{\andER}      [1]{\RightLabel{$\land E2$}\UnaryInfC{#1}  }
\newcommand{\orIL}       [1]{\RightLabel{$\lor I1$} \UnaryInfC{#1}  }
\newcommand{\orIR}       [1]{\RightLabel{$\lor I2$} \UnaryInfC{#1}  }
\newcommand{\orE}        [1]{\RightLabel{$\lor E$}  \TrinaryInfC{#1}}
\newcommand{\arrowI}     [1]{\RightLabel{$\to I$}   \BinaryInfC{#1} }
\newcommand{\arrowE}     [1]{\RightLabel{$\to E$}   \BinaryInfC{#1} }
\newcommand{\notE}       [1]{\RightLabel{$\lnot E$} \BinaryInfC{#1} }
\newcommand{\notI}       [1]{\RightLabel{$\lnot I$} \UnaryInfC{#1}  }
\newcommand{\FE}         [1]{\RightLabel{$\F E$}    \UnaryInfC{#1}  }
\newcommand{\TI}         [1]{\RightLabel{$\T I$}    \UnaryInfC{#1}  }
\newcommand{\LEM}        [1]{\RightLabel{$LEM$}     \UnaryInfC{#1}  }
\newcommand{\DLL}        [1]{\RightLabel{$DL1$}     \UnaryInfC{#1}  }
\newcommand{\DLR}        [1]{\RightLabel{$DL2$}     \UnaryInfC{#1}  }
\newcommand{\notNotE}    [1]{\RightLabel{$\lnot \lnot E$} \UnaryInfC{#1} }
\begin{document}

\maketitle

\pagebreak

\begin{homeworkProblem}
    last week we showed that nand $\odot$ is a universal operator.
    That is, we can write all operators in terms of nand.
    Show that $\to$ is a universal operator by writing $\lnot, \land,$ and $\lor$
    with only $\to$.

    \begin{itemize}
       \item $A \to \bot = \lnot A$
       \item $(A \to \bot) \to B = A \land B$
       \item $(A \to B) \to B = A \lor B$
    \end{itemize}

\end{homeworkProblem}

\begin{homeworkProblem}
   Convert the following to CNF

   \textbf{Part One}\\
   $(c \land a) \lor (b \land c)$ \\
   $(c \lor b) \land (a \lor b) \land (a \lor c)$ \\

   \textbf{Part Two}\\
   $(a \land \lnot a) \lor (b \land \lnot b)$\\
   $\bot$

   \textbf{Part Three}\\
   $a \to (b \equiv c)$\\

   \textbf{Part Four}\\
   $(a \to b) \land (b \to c)$ \\

   \textbf{Part Five}\\
   $\lnot (a \lor b)$ \\

   \textbf{Part Six}\\
   $(a \equiv b) \equiv c$ \\

\end{homeworkProblem}

\begin{homeworkProblem}
   Prove the following: \\

   \textbf{Part One}\\
    $a \lor b \proves b \lor a$: \\

\begin{prooftree}
   \premise{$A \lor B$}
  \assumption{$A$}
  \assumption{$A$}
  \orIR{$B \lor A$}
  \arrowI{$A \to B \lor A$}
   \assumption{$B$}
   \assumption{$B$}
   \orIL{$B \lor A$}
   \arrowI{$B \to B \lor A$}
   \orE{$B \lor A$}
\end{prooftree}

   \textbf{Part Two}\\
    $(a \lor b), \lnot b \proves a$: \\ 

\begin{prooftree}
   \premise{$A \lor B$}
   \assumption{$A$}
   \assumption{$A$}
   \arrowI{$A \to A$}
   \assumption{$B$}
   \assumption{$B$}
   \premise{$\lnot B$}
   \notE{$\bot$}
   \FE{$A$}
   \arrowI{$B \to A$}
   \orE{$A$}
\end{prooftree}

   \textbf{Part Three}\\
    $\lnot a \lor \lnot b \proves \lnot (a \land b)$: \\

\begin{prooftree}
   \premise{$\lnot A \lor \lnot B$}
   \assumption{$\lnot A$}
   \assumption{$A \land B$}
   \assumption{$\lnot A$}
   \assumption{$A \land B$}
   \andER{$A$}
   \notE{$\bot$}
   \arrowI{$A \land B \to \bot$}
   \notI{$\lnot (A \land B)$}
   \arrowI{$\lnot A \to \lnot (A \land B)$}
   \assumption{$\lnot B$}
   \assumption{$A \land B$}
   \assumption{$\lnot B$}
   \assumption{$A \land B$}
   \andEL{$B$}
   \notE{$\bot$}
   \arrowI{$A \land B \to \bot$}
   \notI{$\lnot (A \land B)$}
   \arrowI{$\lnot B \to \lnot (A \land B)$}
   \orE{$\lnot (A \land B)$}
   \
\end{prooftree}

   \textbf{Part Four}\\
    DL1: $\lnot(\lnot a \lor \lnot b) \proves a$

   \textbf{Part Five}\\
    $\lnot(a \land b) \proves \lnot a \lor \lnot b$: \\
          (Hint: you can use the previous problem, and a theorem from class.\\
          \begin{tabular}{ccc}
              \premise{$\lnot(\lnot a \lor \lnot b)$}
              \DLL{$a$}
              \DisplayProof &
              \premise{$\lnot(\lnot a \lor \lnot b)$}
              \DLR{$b$}
              \DisplayProof &
              \premise{$\lnot\lnot a$}
              \notNotE{$a$}
              \DisplayProof \\
          \end{tabular}

\end{homeworkProblem}

\end{document}
