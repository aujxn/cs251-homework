
\documentclass[12pt,a4paper]{article}
\usepackage{bussproofs}
\usepackage{amsmath}

\title{Proof Trees}
\author{Steven Libby}
\date{\vspace{-5ex}}

% these are several symbols that I redefined to make formulas easier to read
\def\land{\wedge}           % /\
\def\lor{\vee}              % \/
\def\lnot{\neg}             % -
\def\liff{\leftrightarrow}  % <->
\def\xor{\oplus}            % (+)
\def\T{\top}                % T
\def\F{\bot}                % _|_
\def\proves{\vdash}         % |-

% these are commands that use the bussproofs package.
% they let you write tree style proofs.
% you can get the package here:  https://www.math.ucsd.edu/~sbuss/ResearchWeb/bussproofs/bussproofs.sty

\newcommand{\premise}    [1]{\AxiomC{#1}}
\newcommand{\assumption} [1]{\AxiomC{[#1]}}
\newcommand{\andI}       [1]{\RightLabel{$\land I$} \BinaryInfC{#1} }
\newcommand{\andEL}      [1]{\RightLabel{$\land E1$}\UnaryInfC{#1}  }
\newcommand{\andER}      [1]{\RightLabel{$\land E2$}\UnaryInfC{#1}  }
\newcommand{\orIL}       [1]{\RightLabel{$\lor I1$} \UnaryInfC{#1}  }
\newcommand{\orIR}       [1]{\RightLabel{$\lor I2$} \UnaryInfC{#1}  }
\newcommand{\orE}        [1]{\RightLabel{$\lor E$}  \TrinaryInfC{#1}}
\newcommand{\arrowI}     [1]{\RightLabel{$\to I$}   \BinaryInfC{#1} }
\newcommand{\arrowE}     [1]{\RightLabel{$\to E$}   \BinaryInfC{#1} }
\newcommand{\notE}       [1]{\RightLabel{$\lnot E$} \BinaryInfC{#1} }
\newcommand{\notI}       [1]{\RightLabel{$\lnot I$} \UnaryInfC{#1}  }
\newcommand{\FE}         [1]{\RightLabel{$\F E$}    \UnaryInfC{#1}  }
\newcommand{\TI}         [1]{\RightLabel{$\T I$}    \UnaryInfC{#1}  }
\newcommand{\LEM}        [1]{\RightLabel{$LEM$}     \UnaryInfC{#1}  }

\begin{document}
\maketitle

This is meant to be a more complete explanation of proof trees.
Proof Trees are a graphical format for formal proofs that eliminate the need to worry about getting references correct.
I'll introduce the concept with several examples.

The basic idea of a proof tree is pretty straightforward.
If you can match all of the premises, then you can derive the conclusion.
This leads to structures that look like the following.
\begin{prooftree}
    \premise{A}
    \UnaryInfC{B}

    \premise{C}
    \premise{D}
    \BinaryInfC{E}

    \premise{A}
    \UnaryInfC{B}
    \premise{A}
    \BinaryInfC{B}

    \TrinaryInfC{F}
\end{prooftree}


Before we go any further, I need to explain a few terms.
A \textbf{premise} is the expression above the bar in an inference rule.
It is either something we were given, something we assumed, or something we derived.
A \textbf{conclusion} is the expression below the bar in an inference rule.
It was derived from premises.
A \textbf{proof tree} is a tree made out of inference rules.
The \textbf{leaves} of the proof tree are anything that isn't under a bar.
A proof tree is \textbf{valid} if all of the leaves are given or assumed, and the premises of inference rules match the definition of the inference rules.
Finally, a formula $A$ \textbf{matches} another formula $B$, if we can replace the variables in $B$ in such a way that $A = B$.

\pagebreak
\noindent
Let's look at an example.
Given the rule
\begin{prooftree}
  \premise{$A$}
  \premise{$A \to B$}
  \arrowE{$B$}
\end{prooftree}
prove:
$A, A \to B, B \to C \proves C$.\\
We can do this with the following proof
\begin{prooftree}
  \premise{$A$}
  \premise{$A \to B$}
  \arrowE{$B$}
  \premise{$B \to C$}
  \arrowE{$C$}
\end{prooftree}

Here $A$, $A \to B$, and $B \to C$ are \textit{premises},
$C$ is the \textit{conclusion}, 
The entire structure is the \textit{proof tree}, and $A$, $A \to B$, and $B \to C$ are \textit{leaves} in the tree
(notice that there's no bar above them).
It's clear that the upper left part of the proof tree \textit{matched} our $\to E$ rule.
It's not as clear, however, that the lower use of $\to E$ matched.\\

\noindent
Matching is very important here, so I'll go over it with some examples.
The simplest example is that any formula matches itself.\\
$A \text{ matches } A$.\\

\noindent
Also any formula can be matched by a variable.\\
$A \text{ matches } V$.\\
$A \land B \text{ matches } V$.\\
$A \to (B \land C) \text{ matches } V$.\\

\noindent
And finally, any formula $A$ matches a formula $B$ if their top level operators are the same, and their subformulas match.
Again, this is easier to see by example.\\
$(A \land B) \to C \text{ matches } D \to C$\\
$A \land B \text{ matches } C \land D$\\
$B \land A \text{ matches } A \land B$\\
$(A \land B) \lor (C \land D) \text{ matches } (A \land B) \lor C$\\

\noindent
It's also important to know when a formula \textit{doesn't} match another one.\\
$A \lor B \text{ doesn't match } A \land B$, because the $\land$ doesn't match $\lor$.\\
$(A \to B) \to C \text{ doesn't match } A \to (B \to C)$.  \\
While $A\to B \text{ matches } A$, $C \text{ doesn't match } B \to C$.\\

\noindent
Wrapping this all up: since we only used the $\to E$ rule, 
all of our leaves were premises, and we matched the definition for every rule, our proof tree is \textit{valid}.

\section{Proof Rules}
Now that we have definitions, we need the inference rules.
These are the rules I've been using for this class, although the ideas work with any set of inference rules.

There are two types of rules here:
Introduction rules, which add an operator (a symbol) into the formula,
and Elimination rules, which take a formula with an operator and remove the operator.
Introduction rules introduce something, and elimination rules eliminate something.\\
 
\begin{tabular}{|ccc|}
  \hline

 & & \\
  \premise{$A \land B$}
  \andEL{$A$}
  \DisplayProof &

  \premise{$A \land B$}
  \andER{$B$}
  \DisplayProof &

  \premise{$A$}
  \premise{$B$}
  \andI{$A \land B$}
  \DisplayProof \\

 & & \\

  \premise{$A$}
  \orIL{$A \lor B$}
  \DisplayProof &

  \premise{$B$}
  \orIR{$A \lor B$}
  \DisplayProof &

  \premise{$A \lor B$}
  \premise{$A \to C$}
  \premise{$B \to C$}
  \orE{$C$}
  \DisplayProof \\

 & & \\

  \premise{$[A]$}
  \premise{$[A]$}
  \noLine
  \UnaryInfC{\vdots}
  \noLine
  \UnaryInfC{$B$}
  \arrowI{$A \to B$}
  \DisplayProof &

  &

  \premise{$A$}
  \premise{$A \to B$}
  \arrowE{$B$}
  \DisplayProof \\
 
 & & \\

  \premise{$A \to \F$}
  \notI{$\lnot A$}
  \DisplayProof &

  \premise{$A$}
  \premise{$\lnot A$}
  \notE{$\F$}
  \DisplayProof & \\

 & & \\

  \premise{$\ $}
  \TI{$\T$}
  \DisplayProof &

  \premise{$\F$}
  \FE{$A$}
  \DisplayProof & 

  \premise{$\ $}
  \LEM{$A \lor \lnot A$}
  \DisplayProof \\
 & & \\
  \hline
\end{tabular}
\\


It is worth convincing yourself that these rules are correct.
It's also worth finding the informal proof strategies that correcpond to these rules.
For example: $\lnot I$ is a proof by contradiction.  We assumed $A$, and proved $\F$, so we know that our assumption of $A$ was wrong.

\section{Proofs}
A proof in propositional logic is a sequence of steps, where each step is an inference rule using conclusions from the previous steps.
In the normal 3-column proofs, this means that you have to keep track of what line on which you reached a conclusion.
Tree proofs eliminate this reference tracking.

Now we'll work through a few simple proofs.
I'll explain how the inference rules are used, 
why they're valid proofs,
and some general strategies for coming up with proofs.
Unfortunately, this is not an algorithm for coming up with proofs.
It can still be tricky to figure out which rules will be the most useful to check.

\section{And Introduction and Elimination}
We'll start off with one of the simpler examples of a proof,
the commutativity of $\land$.
The first thing to notice here is that we always have a \textit{goal} that we're trying to prove.
Here the goal is $B \land A$, and it's good to keep this in mind.
In fact, before we even start, we can write $B \land A$ at the bottom of the proof, since we know we have to prove it.

\begin{prooftree}
  \premise{$?$}
  \premise{$?$}
  \BinaryInfC{$B \land A$}
\end{prooftree}

At this point we could use one of five rules to get to $B \land A$.  $\land I, \land E1, \land E2, \lor E, \to E, \text{and} \F E$.
We can not use any of the others, because their conclusions don't \textit{match} $B \land A$.
We'll start with $\land I$, because that one looks promising.
I know this because I've already done the proof.  Deciding which rules are good rules to check comes with practice.

\begin{prooftree}
  \premise{$B$}
  \premise{$A$}
  \andI{$B \land A$}
\end{prooftree}

Now, we're actually ready to finish the proof.
It turns out that we have everything we need.
All that's left is to get $A$ and $B$ from $A \land B$.
We can do this with $\land E$.

\begin{prooftree}
  \premise{$A \land B$}
  \andER{$B$}
  \premise{$A \land B$}
  \andEL{$A$}
  \andI{$B \land A$}
\end{prooftree}

In this proof we used a couple of strategies that are good to keep in mind.\\
\textbf{destruction-construction:} it's often the case that we can pull apart a formula into component parts with elimination rules.
Then we can construct our new formula with introduction rules.\\
\textbf{working backwards:} Since we know what the end result should be, it's often easier to start with that, and find which rules can apply.\\
\textbf{eliminating bad paths:} Finding a proof for a theorem is a lot like solving a puzzle.  We need to find where all of the pieces fit.
We can also get a lot of information by ruling out several pieces from our search.


\section{Arrow Introduction}
Arrow introduction is usually the rule people have problems with.
It's really not that hard, but you have to be okay with assumptions.
Assumptions work just like premises, but you can only use them in the arrow introduction branch.
That is, if you have, 

\begin{prooftree}
  \assumption{$A$}
  \premise{\vdots}
  \noLine
  \UnaryInfC{$B$}
  \arrowI{$A \to B$}
\end{prooftree}

Then you can use the $[A]$ anywhere above the $\to I$ bar.
At this point $[A]$ works just like a premise.
Also note that $[A]$ is the only thing on the left hand side.
This is just marking that $A$ is the variable that we're assuming to be true.
To show how we can use this rule, let's revisit the proof we did earlier.\\
$A, A \to B, B \to C \proves C$.\\
However, let's make this an implication.\\
$A \to B, B \to C \proves A \to C$.\\

Since, we want to prove $A \to C$, then $\to I$ is probably our best choice.

\begin{prooftree}
  \assumption{$A$}
  \premise{$?$}
  \arrowI{$A \to C$}
\end{prooftree}

Now that we have assumed $A$, we can actually complete the proof just like we did before.
We can use $A$ just like any other premise.
We just need the extra step of assume $A$, and derive $C$.

\begin{prooftree}
  \assumption{$A$}

    \assumption{$A$}
    \premise{$A \to B$}
    \arrowE{$B$}

    \premise{$B \to C$}
    \arrowE{$C$}

  \arrowI{$A \to C$}
\end{prooftree}

\pagebreak
\section{Or Elimination}

The Or Elimination rule can be a little bit tricky.
The rule itself says.
\begin{prooftree}
  \premise{$A \lor B$}
  \premise{$A \to C$}
  \premise{$B \to C$}
  \orE{$C$}
\end{prooftree}

Where did $C$ come from?
What does it mean?
Well, Or Elimination is actually a common proof technique.
It's proof by cases.\\
If we have $A \lor B$, then we have two cases.\\
Case 1: $A$ is true, so we show $A \to C$.\\
Case 2: $B$ is true, so we show $B \to C$.\\
Now, no matter which case is true, we know that we can conclude $C$.

We use this to split up our proof into cases.
We can use this on the following theorem.
$A \lor B, B \to C, (A \land D) \to B, D \proves C$.

We know immediately that our conclusion is going to have to be $C$, so we can start with that.
We also already have $B \to C$, and $A \lor B$, so proof by cases looks like a good idea.

\begin{prooftree}
  \premise{$A \lor B$}

  \premise{$?$}
  \UnaryInfC{$A \to C$}
  \premise{$B \to C$}
  
  \orE{$C$}
\end{prooftree}

Now we need to fill in the $A \to C$ branch.
Remember, to get something of the form $A \to C$, we have to assume $A$, and prove $C$.

\begin{prooftree}
  \premise{$A \lor B$}

  \assumption{$A$}
  \premise{$?$}
  \arrowI{$A \to C$}
  \premise{$B \to C$}
  
  \orE{$C$}
\end{prooftree}

\pagebreak
We don't have anything in our premises that lets us go from $A$ to $C$, so let's look for another route.
We have a $B \to C$.  That look promising; let's try that.

\begin{prooftree}
  \premise{$A \lor B$}

  \assumption{$A$}
  \premise{$?$}
  \UnaryInfC{$B$}
  \premise{$B \to C$}
  \arrowE{$C$}
  \arrowI{$A \to C$}

  \premise{$B \to C$}
  
  \orE{$C$}
\end{prooftree}

We're almost done here.  We just need to prove $B$.
This isn't too hard, since we have $A, D,$ and $A \land D \to B$.

\begin{prooftree}
  \premise{$A \lor B$}

  \assumption{$A$}

  \assumption{$A$}
  \premise{$D$}
  \andI{$A\land D$}
  \premise{$(A \land D) \to B$}
  \arrowE{$B$}

  \premise{$B \to C$}
  \arrowE{$C$}
  \arrowI{$A \to C$}

  \premise{$B \to C$}
  
  \orE{$C$}
\end{prooftree}

You'll notice that proof by cases can grow pretty large.
They also tend to be uncommon outside of contrived situations.
The rules for $\lor I$ introduce a new variable (or expression) that is unrelated to the premise.
This is unlikely to be useful in a proof unless you can introduce a variable related to other premises.
However, proof by cases is very useful for dealing with the law of the excluded middle.

\section{Law of the Excluded Middle}

This is an odd rule to get a feel for.
It can be used to prove some odd results, and it seems like we get an expression coming out of nowhere.
LEM proofs also tend to be some of the trickiest proofs to come up with.
I'd recommend trying this if you can't find another way to prove the result.
It has no premise, and the $A$ in the conclusion can be anything, including $\F$.
The reason that this ends up being consistent is because we actually get a pretty weak result.

\begin{prooftree}
  \premise{$\ $}
  \LEM{$A \lor \lnot A$}
\end{prooftree}

Suppose we tried having the $A$ be $\F$.  Then we would have $\F \lor \lnot \F$, and $\lnot \F$ is $\T$, so it's still fine.
In order to use this rule at all, we need to do a proof by cases.
Let's try the following proof that double negation cancels out.

$\lnot \lnot A \proves A$.

This can be a tricky proof.
It would be entirely reasonable to try to prove this with $\lnot E$ or $\F E$.
However, you'll end up stuck trying to prove it either of these ways.
So let's try a LEM proof.
We know we have to use proof by cases with LEM, so let's get that set up.

\begin{prooftree}

  % LEM branch
  \premise{$\ $}
  \LEM{$A \lor \lnot A$}

  % A -> A branch
  \assumption{$A$}
  \premise{$?$}
  \arrowI{$A \to A$}

  % start !A -> A branch
  \assumption{$\lnot A$}
  \premise{$?$}
  \arrowI{$\lnot A \to A$}

  \orE{$A$}
\end{prooftree}

Ok, so one of these branches is going to be pretty easy to fill in.
We assumed $A$, so we can use that anywhere in that subproof.


\begin{prooftree}

  % LEM branch
  \premise{$\ $}
  \LEM{$A \lor \lnot A$}

  % A -> A branch
  \assumption{$A$}
  \assumption{$A$}
  \arrowI{$A \to A$}

  % start !A -> A branch
  \assumption{$\lnot A$}
  \premise{$?$}
  \arrowI{$\lnot A \to A$}

  \orE{$A$}
\end{prooftree}

The rest of this isn't as clear, but it's not too bad.
Remember we have $\lnot \lnot A$ as a premise, and we just assumed $\lnot A$.
Or, to put that another way, we have $(\lnot A)$ and $\lnot (\lnot A)$.
At this point we can use proof by contradiction ($\lnot E$) to finish the proof.

\begin{prooftree}

  % LEM branch
  \premise{$\ $}
  \LEM{$A \lor \lnot A$}

  % A -> A branch
  \assumption{$A$}
    \assumption{$A$}
  \arrowI{$A \to A$}

  % start !A -> A branch
  \assumption{$\lnot A$}

    \assumption{$\lnot A$}
    \premise{$\lnot \lnot A$}
    \notE{$\F$}

    \FE{$A$}

  \arrowI{$\lnot A \to A$}

  \orE{$A$}
\end{prooftree}

\end{document}
